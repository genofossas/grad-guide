% Document basics
\documentclass[12pt]{article}
\usepackage[margin=1in]{geometry}

% Bibliography packages
\usepackage[
backend=biber,
style=numeric,
]{biblatex}
\addbibresource{ref.bib}


\begin{document}

\begin{titlepage}
    \begin{center}
        \vspace*{1cm}

        \Huge
        \textbf{So, you figured out you want to go to grad school 
                for CS late in your undergrad. What now?}
 
        \large
        \vspace{0.5cm}

        A tired, recently graduated undergrad’s guide to dealing
        with preparing for graduate school applications late in the game.
             
        \vspace{1.5cm}

        \vfill

        \small
        Genoveva Fossas
             
    \end{center}
 \end{titlepage}

\tableofcontents
\pagebreak

\section{Introduction}

Welcome to the guide! I plan to cover a small group of undergraduate research and graduate school related topics, which are fully detailed in the table of contents. My goal is to provide some insight into how I got accepted into graduate school and the way I utilized my time doing undergraduate research. Before I get into the meat of this document, I’ll introduce myself: I’m an undergraduate student that graduated in Spring 2021 from a public university in the USA ranked around 52 on CSRankings\footnote{As of May 2021}. As you might expect, I got a BS in computer science. I was recently accepted into a computer science PhD program that is ranked in the top 20 on CSRankings. It’s important to keep these details about me in mind, as it influences my opinions and what I include in this document. I would also like to fully emphasize that my opinions are my own, and don’t reflect those of my employer or any educational institution I attended or currently attend. 

I don’t expect that this document will be useful to everyone, but I hope to provide some use to those that may be in the same boat. Applying to graduate school stressed me out, and I wish I had more opinions to read and compare. Especially opinions of people that had a pretty mediocre GPA when applying, like myself. This document is going to be an amalgamation of things I've heard from my undergraduate mentor and things I've learned in helpful documents. Everything that isn't from my mentor will be properly cited.

So, you figured out you want a grad degree late in your undergrad. There’s a lot of things to work on and consider from here on out, so we should begin with the most important question:

\subsection{Are You Sure You Want a Graduate Degree?}

You figured out you want the degree, but are you really sure? Many PhD holders cite getting their PhD as a long and arduous process that had incredible peaks but also very demoralizing troughs. Some suggest treating it as a strange job rather than being a student, as very few responsibilities of a PhD student revolve around taking classes. \cite{AzumaPhD} As a PhD student, you are an apprentice academic learning under the guidance of your advisor. You do take a few classes in the beginning, but high marks do not warrant special attention. Classes are meant to impart some basic knowledge and skills and are not a determining factor in your success (outside fulfilling degree requirements). Research success is determined by papers, presentations, products, prototypes, and otherwise disseminating the findings of your research.

This is question is especially important in computer science, as a PhD (and sometimes MS) do not offer a good return on investment if the goal is to boost your salary. A BS is good enough to land you most entry level software development jobs in industry, and work experience will take you much farther when moving up the corporate ladder and applying for other industry jobs. However, obtaining a PhD will open doors to research oriented jobs in academia, industry, and government, so if you're interested in doing research as career a PhD is a good choice. \cite{robertsonwebsite} A MS in certain areas, such as machine learning, may open some doors for you in industry, so you should do more research on your chosen area of interest before committing to a MS.

It took me a while to be wholly sure about getting the PhD, but I found thinking about what jobs I'd like to have really solidified that a PhD is for me. If you think you're sure about getting a PhD, continue reading. Onward and forward!

\section{Graduate School Applications}

Graduate school applications are intimidating for many. This is especially true for first generation college students like myself. Understanding the graduate school application process is the most important step in preparing, so placing the information I gathered about it as the first real section of the guide seemed best. Before I get into details, though, I want to emphasize that what I say here \textbf{will not apply for all graduate programs}. This is especially true for masters degree applications, as I will be primarily addressing PhD programs.

The primary trait that the admission committee want to see in PhD applicants is ability to succeed as a researcher. This can be demonstrated with past published papers and research experience. It is not disastrous to not have research experience, especially if you have great GRE scores, a high GPA, and can clearly demonstrate that you have research potential\footnote{I would still suggest a student in this situation to gain some research experience, as it is a more clear representation of what day-to-day life will be like in a PhD program.}. However, it does improve your application considerably to have research experience. This is especially so for applicants with average GRE scores and GPA. If you have a few semesters left in your undergrad, it is not too late to start undergraduate research. I personally started in the first semester of my junior year. Look at the list of professors in your school and locate one that does research that you are interested in\footnote{More on contacting professors to come in a future section.}.


\subsection{Application Components}

Graduate school applications have some commonalities with undergraduate applications, but there are several additional components not present in undergraduate applications. The components of a graduate application usually include:

\begin{itemize}
    \item Information about the applicant
    \begin{itemize}
        \item Your basic information, earned degrees, work history, transcripts, etc 
    \end{itemize}
    \item Statement of purpose
    \item Recommendation letters
    \item Test scores*
    \item List of potential advisors*
    \item Miscellaneous essays*
\end{itemize}

Items with asterisks may not show up in all applications. Some schools require more information than others, and many also require the same information in different ways. For example, some schools have a focus on community outreach and will ask for additional material that details your volunteering experience. What test scores a program requires may also vary, as some schools no longer accept the GRE. International students may also be required to take additional tests, such as the TOFEL. That is why it's important to read the application instructions closely to make sure you are submitting all the proper information.

\subsubsection{Statement of Purpose}

The statement of purpose is one of the most important parts of your application. The statement of purpose is your opportunity to market yourself and your skill set to the admissions committee and any professors you may be interested in being advised by. There has been a lot written on composing a good statement of purpose, so perusing the greater pool of information on this topic is definitely advised, but I'll address the common recommendations I followed when applying to PhD programs.

The primary information to include in your statement is what degree you're seeking, the general area of research you're interested in, your past research (or related industry) experience, projects you've worked on, and some indication of what lab or professors you're interested in. Avoid telling stories about your experiences with computer science in childhood, as those experiences are generally not indicative of your current skills and some committee members really don't care for them. \cite{mightgradguide,vasansop} Your statement should only be roughly 1 page with a readable font and acceptable margins. I personally used a 11pt serif font with 1 inch margins. Also be sure to check the application for specific formatting directions for your statement, as some schools will require you to double space your statement and give you a larger page limit to account for the extra line space. Given your limited space, it is important to only expand upon relevant details of your experience. \textbf{Most importantly of all, do not use a general statement for all the programs you're applying to.} You should be naming specific professors or labs in your statement and gearing how you discuss your experience to their research interests.

Successfully marketing yourself and your experience is something that comes with practice and looking at examples. It might take a long while to write your first draft as a consequence. The general guidance, however, is to stick to concrete details and provide numerical figures for your work. For instance: what did your work improve, and by how much? Naturally this example will not always apply to all experiences, but being concrete about what your work has contributed and how it makes you a better candidate for a lab is important. If you have industry experience that is not directly related, it might be still marketable as demonstrations of your soft skills. Soft skills include working in a team, ability to discuss and present technical content, and ability to be independent, among many other social skills.

Before submitting your statement, be sure to have several people look over it for you. You can generally point out most awkward phrasing and grammar mistakes if you come back to it after 24 hours, but others will be able to point out sections that are confusing with more ease. The more eyes combing over your statement, the better.

\subsubsection{Recommendation Letters}

Recommendation letters are often considered the most important part of your application. Most programs require 3 letters. Recommendation letters should be from professors, academic staff, or industry professionals. There is a preference for professors to be your letter writers, but many applications do not have any strict requirements on how many of your letter writers must be professors or other academics. In my case, two of my letter writers were professors and the last was an old manager from an internship. Letters should go beyond "this student got an A in my class," so choosing professors you have a close working relationship is best. Your letter writer should be able to expand upon your research potential from their experience with you. It is not disastrous if you have one weak letter, but the other two letters you have should be solid to compensate for the weakness.

As you would expect, the best way to get strong letters is to form strong working relationships with professors. The easiest way to do this is to participate in undergraduate research and interact with the professor(s) overseeing the research, but you can also interact with professors teaching courses by going to office hours and participating in class.

\subsection{Choosing and Contacting Potential Advisors}

As mentioned several times in the above paragraphs, having a set of professors or a lab you're interested in is somewhat important. It makes your statement of purpose easier to tailor. However, it is not strictly required to have this list of professors, especially if you're not exactly sure what area of computer science you're most interested in. I will not discuss this case in depth here since I have little experience with it, but I will defer to a great guide written by many different members of the systers that addresses how to handle this case in your application. \cite{systerssop}

\subsection{Deadlines}

\section{References}

\printbibliography[heading=none]

\end{document}