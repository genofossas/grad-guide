% Document basics
\documentclass[12pt]{article}
\usepackage[margin=1in]{geometry}

% Bibliography packages
\usepackage[
backend=biber,
style=numeric,
]{biblatex}
\addbibresource{ref.bib}


\begin{document}

\begin{titlepage}
    \begin{center}
        \vspace*{1cm}

        \Huge
        \textbf{So, you figured out you want to go to grad school 
                for CS late in your undergrad. What now?}
 
        \large
        \vspace{0.5cm}

        A tired, recently graduated undergrad’s guide to dealing
        with preparing for graduate school applications late in the game.
             
        \vspace{1.5cm}

        \vfill

        \small
        Genoveva Fossas
             
    \end{center}
 \end{titlepage}

\tableofcontents
\pagebreak

\section{Introduction}

Welcome to the guide! I plan to cover a small group of undergraduate research and graduate school related topics, which are fully detailed in the table of contents. My goal is to provide some insight into how I got accepted into graduate school and the way I utilized my time doing undergraduate research. Before I get into the meat of this document, I’ll introduce myself: I’m an undergraduate student that recently graduated (Spring 2021) from a public university in the USA ranked, as of May 2021, around 52 on CSRankings. As you might expect, I got a BS in computer science. I was recently accepted into a computer science PhD program that is ranked in the top 20 on CSRankings. It’s important to keep these details about me in mind, as it influences my opinions and what I include in this document. I would also like to fully emphasize that my opinions are my own, and don’t reflect those of my employer or any educational institution I attended or currently attend. 

I don’t expect that this document will be useful to everyone, but I hope to provide some use to some that may be in the same boat, as this process stressed me out and I wish I had someone else that could provide their opinions on how to proceed. This document is going to be an amalgamation of things I've heard from my undergraduate mentor and things I've learned in helpful documents. Everything that isn't from my mentor will be properly cited.

So, you figured out you want a grad degree late in your undergrad. There’s a lot of things to work on and consider from here on out, so we should begin with the most important question:

\subsection{Are You Sure You Want a Graduate Degree?}

You figured out you want the degree, but are you really sure? Many PhD holders cite getting their PhD as a long and arduous process that had incredible peaks but also very demoralizing troughs. Some suggest treating it as a strange job rather than being a student, as very few responsibilities of a PhD student revolve around taking classes. \cite{AzumaPhD} As a PhD student, you are an apprentice academic learning under the guidance of your advisor. You do take a few classes in the beginning, but high marks do not warrant special attention. Classes are meant to impart some basic knowledge and skills and are not a determining factor in your success (outside fulfilling degree requirements). Research success is determined by papers, presentations, products, prototypes, and otherwise disseminating the findings of your research.

This is question is especially important in computer science, as a PhD (and sometimes MS) do not offer a good return on investment if the goal is to boost your salary. A BS is good enough to land you most entry level software development jobs in industry, and work experience will take you much farther when moving up the corporate ladder and applying for other industry jobs. However, obtaining a PhD will open doors to research oriented jobs in academia, industry, and government, so if you're interested in doing research as career a PhD is a good choice. \cite{robertsonwebsite} A MS in certain areas, such as machine learning, may open some doors for you in industry, so you should do more research on your chosen area of interest before committing to a MS.

It took me a while to be wholly sure about getting the PhD, but I found thinking about what jobs I'd like to have really solidified that a PhD is for me. If you think you're sure about getting a PhD, continue reading. Onward and forward!

\section{Graduate School Applications}

\section{References}

\printbibliography[heading=none]

\end{document}