% Document basics
\documentclass[12pt]{article}
\usepackage[margin=1in]{geometry}

% Bibliography packages
\usepackage[
backend=biber,
style=numeric,
]{biblatex}
\addbibresource{ref.bib}

\usepackage{hyperref}
\hypersetup{
    colorlinks=true,
    allcolors=blue
}

\begin{document}

\begin{titlepage}
    \begin{center}
        \vspace*{1cm}

        \Huge
        \textbf{So, you figured out you want to go to grad school 
                for CS late in your undergrad. What now?}
 
        \large
        \vspace{0.5cm}

        A tired, recently graduated undergrad’s guide to dealing
        with preparing for graduate school applications late in the game.
             
        \vspace{1.5cm}

        \vfill

        \small
        Genoveva Fossas
             
    \end{center}
 \end{titlepage}

\begingroup
\hypersetup{linkcolor=black}
\tableofcontents
\pagebreak
\endgroup

\section{Introduction}

Welcome to the guide! I plan to cover a small group of undergraduate research and graduate school related topics, which are fully detailed in the table of contents. My goal is to provide some insight into how I got accepted into graduate school and the way I utilized my time doing undergraduate research. Before I get into the meat of this document, I’ll introduce myself: I’m an undergraduate student that graduated in Spring 2021 from a public university in the USA ranked around 52 on CSRankings\footnote{As of May 2021}. As you might expect, I got a BS in computer science. I was recently accepted into a computer science PhD program that is ranked in the top 20 on CSRankings. It’s important to keep these details about me in mind, as it influences my opinions and what I include in this document. I would also like to fully emphasize that my opinions are my own, and don’t reflect those of my employer or any educational institution I attended or currently attend. 

I don’t expect that this document will be useful to everyone, but I hope to provide some use to those that may be in the same boat. Applying to graduate school stressed me out, and I wish I had more opinions to read and compare. Especially opinions of people that had a mediocre GPA when applying, like myself. This document is going to be an amalgamation of things I've heard from my undergraduate mentor and things I've learned in helpful documents. Everything that isn't from my mentor will be properly cited. \textbf{Before I get into the thick of things, I want to note that this guide will cover mostly PhD topics.} There will be occasional mentions of masters degrees, but I mostly have experience working with PhD holders that didn't pursue a masters separately.

So, you figured out you want a grad degree late in your undergrad. There’s a lot of things to work on and consider from here on out, so we should begin with the most important question:

\subsection{Are You Sure You Want a Graduate Degree?}

You figured out you want the degree, but are you really sure? Many PhD holders cite getting their PhD as a long and arduous process that had incredible peaks but also very demoralizing troughs. \cite{AzumaPhD} There have been books worth of opinions written on why or why not someone should pursue a PhD, but I'll cover what I read and internalized the most.

\subsubsection{What Getting a PhD Really Means}

Getting a PhD is a very different process from earning your bachelors. Some suggest treating it as a strange job rather than being a student, as very few responsibilities of a PhD student revolve around taking classes. \cite{AzumaPhD} As a PhD student, you are an apprentice academic learning under the guidance of your advisor. You do take a few classes in the beginning, but high marks do not warrant special attention. Classes are meant to impart some basic knowledge and skills and are not a determining factor in your success (outside fulfilling degree requirements). Success for a PhD student is mostly determined by making significant progress towards disseminating your findings. More generally, research success is determined by papers, presentations, products, prototypes, and otherwise disseminating the findings of your research.

\subsubsection{Graduate Degrees in Computer Science}

Determining if you really want a graduate degree is especially important in computer science, as a PhD (and sometimes MS) do not offer a good return on investment if the goal is to boost your salary. A BS is good enough to land you most entry level software development jobs in industry, and work experience will take you much farther when moving up the corporate ladder and applying for other industry jobs. A MS in certain topics (machine learning, artificial intelligence, etc) may be necessary for some industry positions involving that topic, so you should do more research on your chosen area of interest before committing to a MS. Obtaining a PhD, however, will open doors to research oriented jobs in academia, industry, and government, making it a good choice for those that wish to make research their career. \cite{robertsonwebsite} It took me a while to be wholly sure about getting the PhD, but I found thinking about what jobs I'd like to have really solidified that a PhD is for me. If you think you're sure about getting a PhD, continue reading. Onward and forward!

\section{Graduate School Applications}

Graduate school applications are intimidating for many. This is especially true for first generation college students like myself. Understanding the graduate school application process is the most important step in preparing, so placing the information I gathered about it as the first real section of the guide seemed best. Before I get into details, though, I want to emphasize that what I say here \textbf{will not apply for all graduate programs}. This is especially true for masters degree applications, as I will be primarily addressing PhD programs.

The primary trait that the admission committee want to see in PhD applicants is ability to succeed as a researcher. To call back to earlier in this document, research success is determined by published papers, presentations, products, prototypes, and otherwise disseminating the findings of your research. Your potential to be a successful researcher can be demonstrated with published papers (unlikely) or past research experience. It is not disastrous to not have research experience, especially if you have great GRE scores, a high GPA, and can clearly demonstrate that you have a genuine interest in research\footnote{Research experience is still recommended in this case, as it is a better reflection of what grad school will be like. I will not be extensively covering this case, so looking elsewhere may provide better tailored advice.}. However, it does improve your application considerably to have research experience. This is especially so for applicants with average GRE scores and GPA. If you have a few semesters left in your undergrad, it is not too late to start undergraduate research. I personally started in the first semester of my junior year. Look at the list of professors in your school and email one that does research that you are interested in.

\subsection{Application Components}

Graduate school applications have some commonalities with undergraduate applications, but there are several additional components not present in undergraduate applications. The components of a graduate application usually include:

\begin{itemize}
    \item Information about the applicant
    \begin{itemize}
        \item Your basic information, earned degrees, work history, transcripts, etc 
    \end{itemize}
    \item Statement of purpose
    \item Recommendation letters
    \item Test scores*
    \item List of potential advisors*
    \item Miscellaneous essays*
\end{itemize}

Items with asterisks may not show up in all applications. Some schools require more information than others, and many also require the same information in different ways. For example, some schools have a focus on community outreach and will ask for additional material that details your volunteering experience. What test scores a program requires may also vary, as some schools no longer accept the GRE. International students may also be required to take additional tests, such as the TOFEL. That is why it's important to read the application instructions closely to make sure you are submitting all the proper information.

\subsubsection{Statement of Purpose}

The statement of purpose is one of the most important parts of your application. The statement of purpose is your opportunity to market yourself and your skill set to the admissions committee and any professors you may be interested in being advised by. The goal for the statement of purpose is the same as the rest of the application: convince the admissions committee that you have the potential to be successful in research. There has been a lot written on composing a good statement of purpose, so perusing the greater pool of information on this topic is definitely advised, but I'll address the common recommendations I followed when applying to PhD programs.

The primary information to include in your statement is what degree you're seeking, why you're seeking that degree, the general area of research you're interested in, your past research (or related industry) experience, projects you've worked on, and some indication of what lab or professors you're interested in. Successfully marketing yourself is something that comes with practice and looking at examples. The general guidance, however, is to demonstrate that you have the interest, experience, and potential to do research. \cite{mightgradguide} Some things to address when describing your experience is the goal of the project, the core technical challenge, and the key insights into the solution. \cite{mightgradguide} If you have industry experience that is not directly related, it might be still marketable as demonstrations of your soft skills. Soft skills usually refer to a combination of interpersonal skills, communication skills, critical thinking skills, and organization skills. \cite{cotesoftskills} Soft skills include ability to work in a team, problem solve independently, and direct your own work.

Avoid telling stories about your experiences with computer science in childhood, as those experiences are generally not indicative of your current skills and some committee members really don't care for them. \cite{mightgradguide,vasansop} Your statement should only be roughly 1 page with a readable font and acceptable margins. I personally used a 11pt serif font with 1 inch margins. Also be sure to check the application for specific formatting directions for your statement, as some schools will require you to double space your statement and give you a larger page limit to account for the extra line space. Given your limited space, it is important to only expand upon relevant details of your experience. \textbf{Most importantly of all, do not use a general statement for all the programs you're applying to.} You should be naming specific professors or labs in your statement and gearing how you discuss your experience to their research interests.

Before submitting your statement, be sure to have several people look over it for you. You can generally point out most awkward phrasing and grammar mistakes if you come back to it after 24 hours, but others will be able to point out sections that are confusing with more ease. The more eyes combing over your statement, the better.

\subsubsection{Recommendation Letters}

Recommendation letters are often considered the most important part of your application. Most programs require 3 letters. Recommendation letters should be from professors, academic staff, or industry professionals. There is a preference for professors to be your letter writers, but many applications do not have any strict requirements on how many of your letter writers must be professors or other academics. In my case, two of my letter writers were professors and the last was an old manager from an internship. Letters should go beyond "this student got an A in my class," so choosing professors you have a close working relationship with is best. Your letter writer should be able to expand upon your research potential from their experience with you. It is not disastrous if you have one weak letter, but the other two letters you have should be solid to compensate for the weakness.

As you would expect, the best way to get strong letters is to form strong working relationships with professors. The easiest way to do this is to participate in undergraduate research and interact with the professor(s) overseeing the research, but you can also interact with professors teaching courses by going to office hours and participating in class.

\subsubsection{On GPAs and GRE Scores}

As I mentioned, the recommendation letters and statement of purpose are considered the most important part of your application. This is primarily because they are able to more directly comment on your soft and technical skills in application to your potential to be a successful researcher. If you are able to demonstrate you can be successful doing research, especially with prior research experience, mediocre GPA and GRE scores will not significantly hold you back. It is important to note that this may not be the case for higher ranked institutions, as they have far more applicants and will sometimes be strict about GPA or GRE cutoffs to thin the pile of applications they need to go through. 

For transparency, I applied to 5 programs\footnote{This doesn't include my home university at the time, UCF.} during the COVID-19 pandemic with a 3.38 GPA (The GRE requirements were waived due to the pandemic). I was accepted into 2 of them. While having a mediocre GPA will hinder you a little bit, it will not significantly hinder you, so do not be discouraged from applying anyway. Be sure to apply to your dream schools, even if they're significantly high ranked. I avoided applying to the University of Washington and still regret not trying. The worst they can tell you is no.

\subsection{Choosing and Contacting Potential Advisors}

As mentioned several times in the above paragraphs, having a set of professors or a lab you're interested in is somewhat important. It makes your statement of purpose easier to tailor. However, it is not strictly required to have this list of professors, especially if you're not exactly sure what area of computer science you're most interested in. I will not discuss this case in depth here since I have little experience with it, but I will defer to a great guide written by many different members of the systers that addresses how to handle this case in your application. \cite{systerssop}

To locate professors, you may wish to filter the list of schools on \href{https://csrankings.org}{CSRankings} for your interests. This will give you some names to start with, which could lead to you locating a lab web page with several professors doing research in your area of interest. Be sure to look at the professor's (or lab's) most recent publications, as it is a clearer indication of the areas they're working in currently. If you have a mentor with an academic network, you may wish to leverage that to find a potential advisor. Applying to work under a colleague of your current mentor may result in better odds of being accepted, as your mentor's recommendation may carry more weight. Your milage may vary, however; I was not accepted to work under a previous colleague of my mentor.

Before submitting your application, you may wish to email the professors you're interested in so that they keep an eye out for your application. The recommendation for when to reach out to them varies, but the latest I would suggest is one month before you submit your application. \cite{mightgradguide} Mention which degree you're applying for, your research interests, your prior experience, and some projects of theirs that you're interested in. Be sure to keep the email brief and only include relevant details. Commenting intelligently on some of their research is also great to include where possible. Professors receive a lot of generic emails from prospective students, so be sure to write a tailored email for each professor you plan to contact. Professors may also have guidelines for prospective students to follow when emailing to easily pick out which emails aren't generic, so be sure to closely inspect their website for one.

\subsection{Interviews}

So, you've finally submitted your applications and are waiting anxiously. The next thing to look out for is a professor reaching out (usually by email) to request an interview with you. Not all accepted students are interviewed, as conducting interviews is professor and school dependent, but it's important to keep an eye on the email you used to apply anyway.

The contents of an interview is heavily dependent on the professor interviewing you. Many report that they were not asked traditional technical questions, but some professors may do so to ascertain your technical skill. To prepare for this interview, you'll want to read some of the professor's latest work (if you haven't already) and brush up on any technical skills that you think would be relevant (syntax for programming languages, relevant toolchains, etc). Make sure to read through their website and any other publicly available biographical information to answer any simple questions you may have.

It is important to note that while the professor is interviewing your fitness for their lab, you should also be asking questions of them to determine if they're a good fit for your advising needs. Some questions you may want to ask of a prospective advisor include:

\begin{itemize}
    \item What is your advising style?
    \begin{itemize}
        \item How closely do you work with your graduate students?
    \end{itemize}
    \item How big is your lab?
    \begin{itemize}
        \item How big do you envision your lab becoming?
    \end{itemize}
    \item What positions do students from your lab hold after graduation?
    \item Is there funding for graduate students to attend conferences?
    \item Are you okay with your graduate students taking summer internships?
    \item Where does your funding come from?
\end{itemize}

There are many questions that I have not listed here, as students can have numerous combinations of priorities and desires. Combinations that result in far too many questions to include here. A good place to start exploring more questions is a document produced by several students at Colombia University. \cite{colombiaquestions} That document has questions that are geared for visiting after acceptance, but there are several questions on there meant for advisors directly that you can ask while being interviewed. Alongside asking questions of the professor directly, it is worthwhile to contact the professor's current and past graduate students. Not only will graduate students be able to answer questions about general lab culture, they will be able to give you a more candid description of the professor's advising style and demeanor. This is especially so for recently graduated students, as they will hold less reservations about detailing things about the lab or professor that they didn't like.

\section{Undergraduate Research}

Now that the details about the graduate applications process and components are out of the way, I'll be detailing the skills and experiences I found most important for getting involved in and succeeding at undergraduate research.

\subsection{Finding Professors to Work With}

This section may sound familiar to the "Choosing and Contacting Potential Advisors" section due to the title, but there will be very little overlap. While you will want to have overlapping research interests with a potential advisor and potential undergraduate research mentor, choosing an undergraduate research mentor is a more forgiving process. It is much easier to move labs when you're an undergraduate, so you may want to test out several areas of research you have a passing interest in so you're able to get a better feel for your actual areas of interest in the future. 

Much like choosing a graduate advisor, you'll want to give some consideration to what kind of advising style you want out of a research mentor. Most students will want some more hands-on work with the professor, but it's important to consider a less hands-on approach if you already have some background on the research that the professor does. The general consensus is that recently tenured or new tenure-track professors are going to work with you more directly, as they have less obligations than the average senior tenured professor. Senior professors have a larger academic network for you to leverage, however, so if getting access to a network of certain academics or industry professionals is important to you a senior professor may be worth considering. It would be worthwhile to evaluate the lab's ability to support itself without much oversight from the professor in that case, as supportive lab members will be integral to your learning.


\section{References}

\printbibliography[heading=none]

\end{document}